\documentclass[../main.tex]{subfiles}

\begin{document}
\chapter{Primer capitulo}

\lipsum[1]

\section{Entornos}

\teorema{Teorema Fundamental del cálculo}{
    Este teorema es fundamental para el cálculo muestra la estrecha relación entre la integral y la derivada.

    \[
        \frac{d}{dx}\int f(x)\, dx = f(x)  
    \]
}

\demostracion{
    La demostración queda como ejercicio para el lector.
}

\newpage

\corolario{Segundo teorema fundamental del calculo}{
    Sea $F(X)=\int f(x)\, dx$ entonces:
    
    \[\int_a^b f(x)\,dx = F(a)-F(b)\]
}

\nota{Notación}{}{
    Una notación utilizada para representar esto es la siguiente:

    \[
        \int_a^b f(x)\, dx = \left. F(x) \right|_b^a 
    \]
}


\section{Lógica}

\lema{Tercer excluido}{
    \[\vdash_{LP} \alpha \lor \neg \alpha \]
}

\lipsum[1]



\end{document}