\documentclass[../main.tex]{subfiles}

\begin{document}
\chapter{Entornos definidos}

\lipsum[1]


\section{Teoremas}

\teorema{Teorema Fundamental del cálculo}{
    Este teorema es fundamental para el cálculo muestra la estrecha relación entre la integral y la derivada.

    \[
        \frac{d}{dx}\int f(x)\, dx = f(x)  
    \]
}

\section{Corolario}

\corolario{Segundo teorema fundamental del calculo}{
    Sea $F(X)=\int f(x)\, dx$ entonces:
    
    \[\int_a^b f(x)\,dx = F(a)-F(b)\]
}

\section{Nota}

\nota{Notación}{
    Una notación utilizada para representar esto es la siguiente:

    \[
        \int_a^b f(x)\, dx = \left. F(x) \right|_b^a 
    \]
}

\section{Lema}

\lema{Tercer excluido}{
    \[\vdash_{LP} \alpha \lor \neg \alpha \]
}

\section{Preposición}

\preposicion{Una preposición}{
    Esto es una preposicion
}

\section{Definición}

\dfe{Una definición cualquiera}{
    \lipsum[1]
}

\section{Otros entornos}

\ejemplo{
    Esto es un ejemplo

    \solucion{
        Esta es la solución del ejemplo
    }
}

\demostracion{
    La demostración queda como ejercicio para el lector.
}

\section{Listas}

Listas enumeradas

\listo{
    \item Primero item
    \item Segundo item
    \item Tercer item
}

Listas con puntos

\listu{
    \item Primero item
    \item Segundo item
    \item Tercer item
}

\end{document}