\documentclass[../main.tex]{subfiles}

\begin{document}

\chapter{Codigo para entornos}

Se usa el comando \mintinline[bgcolor=grisfondo]{latex}|\teorema| y toma como primer argumento el titulo del teorema y como segundo argumento la descripción del teorema.

\nota{}{
    En general todos los entornos tienen esta estructura
}

\begin{minted}[bgcolor=grisfondo]{latex}
    \teorema{Titulo}{
        El teorema
    }
\end{minted}

\begin{minted}[bgcolor=grisfondo]{latex}
    \corolario{Titulo}{
        Descripcion
    }
\end{minted}

\section{Comandos de los entornos}

\listo{
    \item {\color{primary}\bf Teorema} \mintinline[bgcolor=grisfondo]{latex}|\teorema{titulo}{descripción}|
    \item {\color{primary}\bf Corolario} \mintinline[bgcolor=grisfondo]{latex}|\corolario{titulo}{descripción}|
    \item {\color{primary}\bf Nota} \mintinline[bgcolor=grisfondo]{latex}|\nota{titulo}{descripción}|
    \item {\color{primary}\bf Lema} \mintinline[bgcolor=grisfondo]{latex}|\lema{titulo}{descripción}|
    \item {\color{primary}\bf Preposición} \mintinline[bgcolor=grisfondo]{latex}|\preposicion{titulo}{descripción}|
    \item {\color{primary}\bf Definición} \mintinline[bgcolor=grisfondo]{latex}|\dfe{titulo}{descripción}|
    \item {\color{primary}\bf Ejemplo} \mintinline[bgcolor=grisfondo]{latex}|\ejemplo{descripción}|
    \item {\color{primary}\bf Solución} \mintinline[bgcolor=grisfondo]{latex}|\solucion{descripción}|
    \item {\color{primary}\bf Demostración} \mintinline[bgcolor=grisfondo]{latex}|\demostracion{descripción}|
    \item {\color{primary}\bf Lista enumerada} \mintinline[bgcolor=grisfondo]{latex}|\listo{\item item}|
    \item {\color{primary}\bf Lista} \mintinline[bgcolor=grisfondo]{latex}|\listu{\item item}|
}

\end{document}