\documentclass{report}
% Configuración
\usepackage[a4paper]{geometry}
\usepackage{fancyhdr}
\usepackage{graphicx, wrapfig, subcaption, setspace, booktabs}
\usepackage[T1]{fontenc}
\usepackage[font=small, labelfont=bf]{caption}
\usepackage[protrusion=true, expansion=true]{microtype}
\usepackage{xcolor}
\usepackage[spanish]{babel}
% Matemáticas
\usepackage{amsmath,amssymb,amsfonts,latexsym,cancel,amsthm}
\usepackage{xparse}
\usepackage{helvet}     % Fuente del documento
\renewcommand{\familydefault}{\sfdefault}

% Colores para las urls
\usepackage[colorlinks=true]{hyperref}
\hypersetup{
    colorlinks=true,
    linkcolor=black,
    filecolor=magenta,      
    urlcolor=blue,
}

\usepackage[utf8]{inputenc}
\usepackage{sectsty}
\usepackage{url, lipsum}
\usepackage{tabularx}
\usepackage{float}


\onehalfspacing
\setcounter{tocdepth}{5}
\setcounter{secnumdepth}{5}

\pagestyle{fancy}
\fancyhf{}
\setlength\headheight{15pt}
\fancyhead[R]{Title}
\fancyfoot[R]{\thepage}

% Formato de los capítulos
\usepackage{titlesec, blindtext, color}
\definecolor{gray75}{gray}{0.75}
\newcommand{\hsp}{\hspace{20pt}}
\titleformat{\chapter}[hang]{\Huge\bfseries}{\thechapter\hsp\textcolor{gray75}{|}\hsp}{0pt}{\Huge\bfseries}

\usepackage[a4paper]{geometry}
\usepackage{graphicx, wrapfig, subcaption, setspace, booktabs}
\usepackage[T1]{fontenc}
\usepackage[spanish]{babel}
\usepackage{arev}
\usepackage[scaled]{helvet}     % Fuente del documento
\renewcommand{\familydefault}{\sfdefault}
\usepackage[utf8]{inputenc}
\usepackage{url, lipsum}
\usepackage{tabularx}

% Puedes cambiar el color principal
% \definecolor{primary}{HTML}{}

\begin{document}

\pagestyle{empty}
\portada{NOTAS DE X TEMA}{Plantilla de notas}{Autor}
\newpage

\tableofcontents

\newpage

\chapter{Entornos definidos}

\lipsum


\section{Teoremas}

\teorema{Teorema Fundamental del cálculo}{
    Este teorema es fundamental para el cálculo muestra la estrecha relación entre la integral y la derivada.

    \[
        \frac{d}{dx}\int f(x)\, dx = f(x)
    \]
}

\section{Corolario}

\corolario{Segundo teorema fundamental del calculo}{
    Sea $F(X)=\int f(x)\, dx$ entonces:

    \[\int_a^b f(x)\,dx = F(a)-F(b)\]
}

\section{Nota}

\nota{Notación}{
    Una notación utilizada para representar esto es la siguiente:

    \[
        \int_a^b f(x)\, dx = \left. F(x) \right|_b^a
    \]
}

\section{Lema}

\lema{Tercer excluido}{
    \[\vdash_{LP} \alpha \lor \neg \alpha \]
}

\section{Preposición}

\preposicion{Una preposición}{
    Esto es una preposicion
}

\section{Definición}

\dfe{Una definición cualquiera}{
    \lipsum[1]
}

\section{Otros entornos}

\ejemplo{
    Esto es un ejemplo

    \solucion{
        Esta es la solución del ejemplo
    }
}

\demostracion{
    La demostración queda como ejercicio para el lector.
}

\section{Listas}

Listas enumeradas

\listo{
    \item Primero item
    \item Segundo item
    \item Tercer item
}

Listas con puntos

\listu{
    \item Primero item
    \item Segundo item
    \item Tercer item
}

\newpage

\chapter{Hola como estas}

\lipsum[2]

\section{Esto es una sección}

\lipsum[1]

\section{Otro capitulo}

\lipsum[1]
\subsection{Hola como estas}
\subsection*{No lo se}

\fancybox{primary}{Prubea de nueva caja}{
    \lipsum[1]
}
\end{document}