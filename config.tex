%----------------
%   Importaciones
%----------------
\usepackage[tmargin=2cm,rmargin=1in,lmargin=1in,margin=0.85in,bmargin=2cm,footskip=.2in]{geometry}
\usepackage{graphicx, wrapfig, subcaption, setspace, booktabs}
\usepackage[T1]{fontenc}
\usepackage[spanish]{babel}
\usepackage{xcolor}
\usepackage[scaled]{helvet}     % Fuente del documento
\renewcommand{\familydefault}{\sfdefault}
\usepackage{amsmath,amssymb,amsfonts,latexsym,cancel,amsthm,mathtools}
\usepackage[utf8]{inputenc}
\usepackage{url, lipsum}
\usepackage{tabularx}
\usepackage{xparse}
\usepackage{sectsty}

% \setlength{\parindent}{0cm}
% \setlength{\parskip}{5pt}

%-----------------
%   Colores
%-----------------
\definecolor{mytheorembg}{HTML}{F2F2F9}
\definecolor{mylenmabg}{HTML}{FFFAF8}
\definecolor{mylenmafr}{HTML}{983b0f}
\definecolor{mypropbg}{HTML}{f2fbfc}
\definecolor{mypropfr}{HTML}{191971}
\definecolor{myp}{RGB}{197, 92, 212}
\definecolor{primary}{HTML}{207ba5}    % Color principal
\definecolor{migris}{RGB}{17, 17, 17}
\definecolor{grisfondo}{RGB}{249, 249, 249}


%----------------
%   Colores para
%   Las urls
%----------------
\usepackage[colorlinks=true]{hyperref}
\hypersetup{
    colorlinks=true,
    linkcolor=black,
    filecolor=magenta,
    urlcolor=blue,
}

%----------------
%   Cajas
%----------------

\usepackage[most]{tcolorbox}

\tcbuselibrary{theorems,skins,hooks}
\NewDocumentCommand\caja{m O{\Large #1} O{grisfondo} O{primary} O{number within=chapter}}
{
    \newtcbtheorem[#5]{#1}{\large #2}
    {%
        enhanced
        ,breakable
        ,colback = #3
        ,frame hidden
        ,boxrule = 0sp
        ,borderline west = {2pt}{0pt}{#4}
        ,sharp corners
        ,detach title
        ,before upper = \tcbtitle\par\smallskip
        ,coltitle = #4
        ,fonttitle = \bfseries\sffamily
        % ,description font = \mdseries
        ,separator sign none
        ,segmentation style={solid, #4}
    }
    {th}
}

\caja{Corolario}[Corolario][myp!10][myp!85!black]
\caja{Lema}[Lema][mylenmabg][mylenmafr]
\caja{Prepo}[Preposición][mypropbg][mypropfr]
\caja{defi}[Definición][primary!7][primary]
\caja{Nota}[Nota][grisfondo][migris][no counter]
\caja{Teorema}[Teorema][primary!7][primary]

%---------------
%   Comandos
%---------------
\newcommand{\teorema}[2]{\begin{Teorema}{#1}{}#2\end{Teorema}}
\newcommand{\corolario}[2]{\begin{Corolario}{#1}{}#2\end{Corolario}}
\newcommand{\lema}[2]{\begin{Lema}{#1}{}#2\end{Lema}}
\newcommand{\preposicion}[2]{\begin{Prepo}{#1}{}#2\end{Prepo}}
\newcommand{\nota}[2]{\begin{Nota}{#1}{}#2\end{Nota}}
\newcommand{\dfe}[2]{\begin{defi}{#1}{}#2\end{defi}}
\newcommand{\demostracion}[1]{\begin{proof}[\color{primary}\textbf{Demostración.}] #1 \end{proof}}

\theoremstyle{definition}
\newtheorem*{ejem}{\color{primary}Ejemplo}
\newcommand{\ejemplo}[1]{\begin{ejem}#1\end{ejem}}

\theoremstyle{definition}
\newtheorem*{solu}{\color{primary}Solución}
\newcommand{\solucion}[1]{\begin{solu}#1\end{solu}}

%---------------
%   Listas
%---------------
\usepackage{tikz}

\usepackage{enumitem}

\newcommand{\cnumero}[2]{
    \tikz[baseline=(myanchor.base)]
    \node[minimum size=0.2cm,circle,
        inner sep=1pt,draw, #2,thick,fill=#2](myanchor)
    {\color{white}\bfseries\fontsize{8}{8}#1};}

\newcommand*{\itembolasazules}[1]{\protect\cnumero{#1}{primary}}

\newcommand{\listo}[1]{
    \begin{enumerate}[label=\itembolasazules{\arabic*}]
        #1
    \end{enumerate}
}

\newcommand{\listu}[1]{
    \begin{itemize}[label=$\color{primary} \bullet$]
        #1
    \end{itemize}
}

%-------------------------
% Tabla de Contenidos
%-------------------------

\usepackage{blindtext}
\usepackage{framed}
\usepackage{titletoc}
\usepackage{etoolbox}

\patchcmd{\tableofcontents}{\contentsname}{\sffamily\contentsname}{}{}

\renewenvironment{leftbar}
{\def\FrameCommand{\hspace{6em}%
        {\color{primary}\vrule width 2pt depth 6pt}\hspace{1em}}%
    \MakeFramed{\parshape 1 0cm \dimexpr\textwidth-6em\relax\FrameRestore}\vskip2pt%
}
{\endMakeFramed}

\titlecontents{chapter}[0em]
{\vspace*{2\baselineskip}}
{\parbox{4.5em}{%
        \hfill\Huge\sffamily\bfseries\color{primary}\thecontentslabel}%
    \vspace*{-2.3\baselineskip}\leftbar\textbf{\color{primary}\small\chaptername~\thecontentslabel}\\\sffamily
}{}{\endleftbar}

\titlecontents{section}[8.4em]
{\sffamily\contentslabel{3em}}{}{}
{\hspace{0.5em}\nobreak\itshape\color{primary}\contentspage}

\titlecontents{subsection}[8.4em]
{\sffamily\contentslabel{3em}}{}{}
{\hspace{0.5em}\nobreak\itshape\color{primary}\contentspage}

%-----------------------------
%   Formato de los capitulos
%-----------------------------

%==================
% Capitulos
%==================
\newtcolorbox{titlecolorbox}[1]{ %the box around chapter
    coltext=white,
    colframe=primary,
    colback=primary,
    boxrule=0pt,
    arc=0pt,
    notitle,
    width=4.8em,
    height=2.4ex,
    before=\hfill
}

\usepackage[explicit]{titlesec}

\makeatletter
\let\old@rule\@rule
\def\@rule[#1]#2#3{\textcolor{primary}{\old@rule[#1]{#2}{#3}}}
\makeatother

\titleformat{\chapter}[display]
{\sffamily\Huge}
{}
{0pt}
{\begin{titlecolorbox}{}
        {\large\sffamily\MakeUppercase{\chaptername}}
    \end{titlecolorbox}
    \vspace*{-4.19ex}\noindent\rule{\textwidth}{0.4pt}
    \parbox[b]{\dimexpr\textwidth-4.8em\relax}{\raggedright\MakeUppercase{#1}}{\hfill\fontsize{70}{60}\selectfont{\color{primary}\thechapter}}
}
[]

\titleformat{name=\chapter,numberless}[display]
{\sffamily\Huge}
{}
{0pt}
{
    \vspace*{-4.19ex}\noindent\rule{\textwidth}{0.4pt}
    \parbox[b]{\dimexpr\textwidth-4.8em\relax}{\raggedright\MakeUppercase{#1}}
}
[]

%==============
% Secciones
%==============

\titleformat{\section}[hang]{\Large\bfseries\sffamily}%
{\rlap{\color{primary}\rule[-6pt]{\textwidth}{1.2pt}}\colorbox{primary}{%
        \raisebox{0pt}[13pt][3pt]{ \makebox[60pt]{% height, width
                \fontfamily{phv}\selectfont\color{white}{\thesection}}
        }}}%
{15pt}%
{ \color{primary}#1
    %
}
\titlespacing*{\section}{0pt}{3mm}{5mm}

%================
% Sub secciones
%================
\subsectionfont{\Large\color{primary}}

%---------------------
% Portada
%---------------------
\usetikzlibrary{ shapes.geometric }
\usetikzlibrary{calc}
\usepackage{anyfontsize}
\newcommand{\portada}[3]{
    \begin{tikzpicture}[remember picture,overlay]
        %%%%%%%%%%%%%%%%%%%% Background %%%%%%%%%%%%%%%%%%%%%%%%
        \fill[primary] (current page.south west) rectangle (current page.north east);


        \foreach \i in {2.5,...,22}
            {
                \node[rounded corners,primary!60,draw,regular polygon,regular polygon sides=6, minimum size=\i cm,ultra thick] at ($(current page.west)+(2.5,-5)$) {} ;
            }

        %%%%%%%%%%%%%%%%%%%% Background Polygon %%%%%%%%%%%%%%%%%%%% 
        \foreach \i in {0.5,...,22}
            {
                \node[rounded corners,primary!60,draw,regular polygon,regular polygon sides=6, minimum size=\i cm,ultra thick] at ($(current page.north west)+(2.5,0)$) {} ;
            }

        \foreach \i in {0.5,...,22}
            {
                \node[rounded corners,primary!90,draw,regular polygon,regular polygon sides=6, minimum size=\i cm,ultra thick] at ($(current page.north east)+(0,-9.5)$) {} ;
            }


        \foreach \i in {21,...,6}
            {
                \node[primary!85,rounded corners,draw,regular polygon,regular polygon sides=6, minimum size=\i cm,ultra thick] at ($(current page.south east)+(-0.2,-0.45)$) {} ;
            }


        %%%%%%%%%%%%%%%%%%%% Title of the Report %%%%%%%%%%%%%%%%%%%% 
        \node[left,primary!5,minimum width=0.625*\paperwidth,minimum height=3cm, rounded corners] at ($(current page.north east)+(0,-9.5)$)
        {
            {\fontsize{25}{30} \selectfont \bfseries #1}
        };

        %%%%%%%%%%%%%%%%%%%% Subtitle %%%%%%%%%%%%%%%%%%%% 
        \node[left,primary!10,minimum width=0.625*\paperwidth,minimum height=2cm, rounded corners] at ($(current page.north east)+(0,-11)$)
        {
            {\huge \textit{#2}}
        };

        %%%%%%%%%%%%%%%%%%%% Author Name %%%%%%%%%%%%%%%%%%%% 
        \node[left,primary!5,minimum width=0.625*\paperwidth,minimum height=2cm, rounded corners] at ($(current page.north east)+(0,-13)$)
        {
            {\Large \textsc{#3}}
        };

        %%%%%%%%%%%%%%%%%%%% Year %%%%%%%%%%%%%%%%%%%% 
        \node[rounded corners,fill=primary!70,text =primary!5,regular polygon,regular polygon sides=6, minimum size=2.5 cm,inner sep=0,ultra thick] at ($(current page.west)+(2.5,-5)$) {\LARGE \bfseries \the\year{}};

    \end{tikzpicture}
}