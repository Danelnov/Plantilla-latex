%----------------
%   Importaciones
%----------------

\usepackage[tmargin=2cm,rmargin=1in,lmargin=1in,margin=0.85in,bmargin=4cm,footskip=.7in]{geometry}
\usepackage{graphicx, wrapfig, subcaption, setspace, booktabs}
\usepackage[T1]{fontenc}
\usepackage[spanish]{babel}
\usepackage{xcolor}
\usepackage[protrusion=true, expansion=true]{microtype}
\usepackage{xparse}
\usepackage[scaled]{helvet}     % Fuente del documento
\renewcommand{\familydefault}{\sfdefault}
% Matemáticas
\usepackage{amsmath,amssymb,amsfonts,latexsym,cancel,amsthm}
\usepackage[utf8]{inputenc}
% \usepackage{sectsty}    % Cambiar el estilo de cualquier cabecera
\usepackage{url, lipsum}
\usepackage{tabularx}
\usepackage{float}
% Colores para codigo
\usepackage{minted}

%-----------------
%   Colores
%-----------------
\definecolor{myg}{RGB}{56, 140, 70}
\definecolor{myb}{RGB}{45, 111, 177}
\definecolor{myr}{RGB}{199, 68, 64}
\definecolor{mytheorembg}{HTML}{F2F2F9}
\definecolor{mytheoremfr}{HTML}{00007B}
\definecolor{mylenmabg}{HTML}{FFFAF8}
\definecolor{mylenmafr}{HTML}{983b0f}
\definecolor{mypropbg}{HTML}{f2fbfc}
\definecolor{mypropfr}{HTML}{191971}
\definecolor{myp}{RGB}{197, 92, 212}
\definecolor{}{HTML}{2C3338}
\definecolor{primary}{RGB}{32, 123, 165}    % Color principal
\definecolor{migris}{RGB}{17, 17, 17}
\definecolor{grisfondo}{RGB}{239, 239, 239}


%----------------
%   Colores para
%   Las urls
%----------------
\usepackage[colorlinks=true]{hyperref}
\hypersetup{
    colorlinks=true,
    linkcolor=black,
    filecolor=magenta,
    urlcolor=blue,
}


%--------------
%   Encabezado
%--------------
% \onehalfspacing
% \setcounter{tocdepth}{5}
% \setcounter{secnumdepth}{5}

%----------------
%   Cajas
%----------------

\usepackage[most]{tcolorbox}

\setlength{\parindent}{1cm}

%================================
% THEOREM BOX
%================================

\tcbuselibrary{theorems,skins,hooks}
\newtcbtheorem[number within=chapter]{Teorema}{\large Teorema}
{%
    enhanced,
    breakable,
    colback = mytheorembg,
    frame hidden,
    boxrule = 0sp,
    borderline west = {2pt}{0pt}{primary},
    sharp corners,
    detach title,
    fonttitle = \bfseries\sffamily,
    before upper = \tcbtitle\par\smallskip,
    coltitle = primary,
    separator sign none,
    segmentation style={solid, primary},
}
{th}

\tcbuselibrary{theorems,skins,hooks}
\newtcolorbox{Theoremcon}
{%
    enhanced
    ,breakable
    ,colback = mytheorembg
    ,frame hidden
    ,boxrule = 0sp
    ,borderline west = {2pt}{0pt}{primary}
    ,sharp corners
    ,description font = \mdseries
    ,separator sign none
}

%================================
% Corollery
%================================
\tcbuselibrary{theorems,skins,hooks}
\newtcbtheorem[number within=chapter]{Corolario}{\large Corolario}
{%
    enhanced
    ,breakable
    ,colback = myp!10
    ,frame hidden
    ,boxrule = 0sp
    ,borderline west = {2pt}{0pt}{myp!85!black}
    ,sharp corners
    ,detach title
    ,before upper = \tcbtitle\par\smallskip
    ,coltitle = myp!85!black
    ,fonttitle = \bfseries\sffamily
    ,description font = \mdseries
    ,separator sign none
    ,segmentation style={solid, myp!85!black}
}
{th}


%================================
% LENMA
%================================

\tcbuselibrary{theorems,skins,hooks}
\newtcbtheorem[number within=chapter]{Lema}{\large Lema}
{%
    enhanced,
    breakable,
    colback = mylenmabg,
    frame hidden,
    boxrule = 0sp,
    borderline west = {2pt}{0pt}{mylenmafr},
    sharp corners,
    detach title,
    before upper = \tcbtitle\par\smallskip,
    coltitle = mylenmafr,
    fonttitle = \bfseries\sffamily,
    description font = \mdseries,
    separator sign none,
    segmentation style={solid, mylenmafr},
}
{th}

%================================
% PROPOSITION
%================================

\tcbuselibrary{theorems,skins,hooks}
\newtcbtheorem[number within=chapter]{Prepo}{\large Preposición}
{%
    enhanced,
    breakable,
    colback = mypropbg,
    frame hidden,
    boxrule = 0sp,
    borderline west = {2pt}{0pt}{mypropfr},
    sharp corners,
    detach title,
    before upper = \tcbtitle\par\smallskip,
    coltitle = mypropfr,
    fonttitle = \bfseries\sffamily,
    description font = \mdseries,
    separator sign none,
    segmentation style={solid, mypropfr},
}
{th}


%================================
% Definición
%================================

\tcbuselibrary{theorems,skins,hooks}
\newtcbtheorem[number within=chapter]{defi}{\large Definición}
{%
    enhanced
    ,breakable
    ,colback = primary!10
    ,frame hidden
    ,boxrule = 0sp
    ,borderline west = {2pt}{0pt}{primary}
    ,sharp corners
    ,detach title
    ,before upper = \tcbtitle\par\smallskip
    ,coltitle = primary!85!black
    ,fonttitle = \bfseries\sffamily
    ,description font = \mdseries
    ,separator sign none
    ,segmentation style={solid, primary!85!black}
}
{th}

%==============
%   Nota
%==============

\tcbuselibrary{theorems,skins,hooks}
\newtcbtheorem[no counter]{Nota}{\large Nota}
{%
    enhanced,
    breakable,
    colback = white,
    frame hidden,
    boxrule = 0sp,
    borderline west = {2pt}{0pt}{migris},
    sharp corners,
    detach title,
    before upper = \tcbtitle\par\smallskip,
    coltitle = migris,
    fonttitle = \bfseries\sffamily,
    description font = \mdseries,
    separator sign none,
    segmentation style={solid, migris},
}
{th}

%---------------
%   Comandos
%---------------
\newcommand{\teorema}[2]{\begin{Teorema}{#1}{}#2\end{Teorema}}
\newcommand{\corolario}[2]{\begin{Corolario}{#1}{}#2\end{Corolario}}
\newcommand{\lema}[2]{\begin{Lema}{#1}{}#2\end{Lema}}
\newcommand{\preposicion}[2]{\begin{Prepo}{#1}{}#2\end{Prepo}}
\newcommand{\nota}[2]{\begin{Nota}{#1}{}#2\end{Nota}}
\newcommand{\dfe}[2]{\begin{defi}{#1}{}#2\end{defi}}
\newcommand{\demostracion}[1]{\begin{proof}[\color{primary}\textbf{Demostración.}] #1 \end{proof}}

\theoremstyle{definition}
\newtheorem*{ejem}{\color{primary}Ejemplo}
\newcommand{\ejemplo}[1]{\begin{ejem}#1\end{ejem}}

\theoremstyle{definition}
\newtheorem*{solu}{\color{primary}Solución}
\newcommand{\solucion}[1]{\begin{solu}#1\end{solu}}

%---------------
%   Listas
%---------------
\usepackage{tikz}

\usepackage{enumitem}

\newcommand{\cnumero}[2]{
    \tikz[baseline=(myanchor.base)]
    \node[minimum size=0.2cm,circle,
        inner sep=1pt,draw, #2,thick,fill=#2](myanchor)
    {\color{white}\bfseries\fontsize{8}{8}#1};}

\newcommand*{\itembolasazules}[1]{\protect\cnumero{#1}{primary}}

\newcommand{\listo}[1]{
    \begin{enumerate}[label=\itembolasazules{\arabic*}]
        #1
    \end{enumerate}
}

\newcommand{\listu}[1]{
    \begin{itemize}[label=$\color{primary} \bullet$]
        #1
    \end{itemize}
}


%-----------
%   Codigo
%-----------
% Caja de codigo
\tcbuselibrary{theorems,skins,hooks}
\newtcbtheorem{Codigo}{}
{%
    enhanced,
    breakable,
    colback = grisfondo,
    frame hidden,
    boxrule = 0sp,
    borderline west = {2pt}{0pt}{primary},
    sharp corners,
    detach title,
    % before upper = \tcbtitle\par\smallskip,
    coltitle = primary,
    fonttitle = \bfseries\sffamily,
    description font = \mdseries,
    separator sign none,
    segmentation style={solid, primary},
}
{th}

\newcommand{\codigo}[2]{
    \begin{Codigo}{#1}{}
        #2
    \end{Codigo}
}

%-------------------------
%   Tabla de contenidos
%-------------------------
\usepackage{titlesec, blindtext}
\usepackage{framed}
\usepackage{titletoc}
\usepackage{etoolbox}
\usepackage{lmodern}


\patchcmd{\tableofcontents}{\contentsname}{\sffamily\contentsname}{}{}

\renewenvironment{leftbar}
{\def\FrameCommand{\hspace{6em}%
        {\color{primary}\vrule width 2pt depth 6pt}\hspace{1em}}%
    \MakeFramed{\parshape 1 0cm \dimexpr\textwidth-6em\relax\FrameRestore}\vskip2pt%
}
{\endMakeFramed}

\titlecontents{chapter}[0em]
    {\vspace*{2\baselineskip}}
    {\parbox{4.5em}{%
        \hfill\Huge\sffamily\bfseries\color{primary}\thecontentspage}%
        \vspace*{-2.3\baselineskip}\leftbar\textsc{\small\chaptername~\thecontentslabel}
        \\\sffamily
    }
    {}{\endleftbar}

%-----------------------------
%   Formato de los capitulos
%-----------------------------
\newcommand{\hsp}{\hspace{20pt}}
\titleformat{\chapter}[hang]
    {\Huge\bfseries}
    {\thechapter\hsp\textcolor{primary}{|}\hsp}
    {0pt}{\Huge\bfseries}